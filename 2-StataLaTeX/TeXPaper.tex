\documentclass[11pt]{article} % use larger type; default would be 10pt
\usepackage[utf8]{inputenc} % set input encoding (not needed with XeLaTeX)
\usepackage{pslatex}
\usepackage{standalone} %required by outreg2 in Stata
\usepackage{dcolumn} %required by stargazer in R

\title{Your Paper That's So Transparent And Reproducible You'll Get a Tenure Track Position}
\author{Garret Christensen}
%\date{} % Activate to display a given date or no date (if empty),
         % otherwise the current date is printed 

\begin{document}
\input{./StataScalarList.tex}
%Just load the file full of scalars at the beginning
%Call them later when you need them
\maketitle

\section{Wait for it}

This is where I introduce you to my amazingness.
I'm going to do a little thing where I include numerical output from Stata.

The mean of the price variable is $\meanprice$ and its standard deviation (rounded to one decimal point) is $\stddevprice$

\subsection{Wait for it.}

More text, and ... a table! 

\begin{table}
\caption{Made Automatically in Stata}
\input{outputS.tex} %created by StataLaTeX.do
\end{table}
%input the table you generated in Stata with outreg2 or estout
%This is a relatively simple 2-click workflow
%One click in Stata to run all your code and generate the results
%and a second click to compile and include those results into your paper.

%The concern is that with a lot of tables, you can forget which script file generated which table. Hence the comment above next to the input statement.



\end{document}
